%!TEX program = xelatex
%!TEX option = -shell-escape -8bit
% Homework List File
\documentclass[a4paper, 11.5pt]{article}

\usepackage[margin={2cm,1.9cm},includefoot]{geometry}
\usepackage{amsmath} % Package for math formula, symbols, etc
\usepackage{amsfonts} % Package for extra math font package
\usepackage{tikz} % Package for vector graphic drawing
\usepackage{csquotes} % Package for qouted fragment
\usepackage{listings} % Package for code fragment
\usepackage{lmodern}
  \usepackage{fontspec}
  \usepackage{hyperref} % use this to support hyper-liniks
  \hypersetup{
    colorlinks=true,
    linkcolor=gray,
    filecolor=magenta,
    urlcolor=gray,
  }
  \urlstyle{same}
  \usepackage{xeCJK}
  \usepackage[skins,breakable]{tcolorbox}
  \usepackage{minted}
  \usemintedstyle{bw} % use bw style since the asm code can't be well-highlighted
  \usepackage{xcolor}
  \renewcommand\theFancyVerbLine{\textcolor{gray}{\normalsize\arabic{FancyVerbLine}}}
\usepackage{multicol}
\usepackage{lineno}
\setlength{\columnsep}{0.5cm}
\usepackage{enumitem}
\setlist[itemize]{leftmargin=*}
\usepackage{indentfirst}
\setlength{\parskip}{1.5pt}
\setlength{\parindent}{1em}
% This line requires the specified font available in system path
\setmainfont{Times New Roman}
\setmonofont{Monaco}

% Roman number
\makeatletter
\newcommand*{\rom}[1]{\expandafter\@slowromancap\romannumeral #1@}
\makeatother

\title{Term Project: Yet Another SAT Solver (YaSat) \\ Milestone II}
\author{喬波\\0540017\\}
\date{Nov. 23th, 2016}

\begin{document}
% \fontfamily{qpl}\selectfont

  % Title
  % \maketitle
  \begin{flushleft}
    \ \\ \ \\
    \large{\em Practice/Real-Life Applications of Computational Algorithms} \\ \ 
  \end{flushleft}

  \begin{center}
    \huge{Term Project: Yet Another SAT Solver (YaSat) }\\
    \LARGE{Milestone \rom{2}}
  \end{center}
  \begin{center}
    \ \\
    喬波\\
    0540017\\
    Dec. 23th, 2016\\ \ \\ \ \\
  \end{center}

  \begin{multicols}{2}

  The term project: Yet Another SAT Solver (YaSat) is to build a program to solve boolean satisfiability (SAT) problem. More specifically, the project is to implement a program 
  that processes a set of boolean clause consisting of boolean literals and represented in 
  conjunctive normal form (CNF), then gives the result that states whether the SAT problem is
  satisfiable and for the satisfiable problem, gives a vaild solution. 

  The simplest idea of implementation is to enumerate all possible assignment combination 
  of the each boolean literal, while this is impractical accounting to its complexity. 
  However, by applying some kinds of processing techniques, a lot of unnecessory attempt can be 
  avoided, thus the complexity will become acceptable, for generall cases. This term project is 
  to build a solver that adopts some techniques to achieve a performance as high as possible.

  This project is relatively complex, and several milestones are set to simplify the control of 
  implementation. Currently implemented version of the project is in the milestone 2, which is the 
  central topic of this report. 

  This report will cover the new features added in the milestone 2, changes to its basic structure, 
  the compiling options and the usage of the program, difficulties and problems during this stage of implementation and the plan for the next milestone.

  \section{New Features}
    At the milestone 2, the work is mainly about the conflict driven clause learning and non-chonological backtracking.

    In milestone 1, the basic BCP and 2 literal watching is implemented. In this milestone,
    the conflict driven clause learning (CDCL) and non-chonological backtracking is implemented 
    by adopting the First Unique Implication Point (FirstUIP) method. Since the clause learning 
    method introduced new clauses into the existing clause database, the original data structure 
    based on array with fixed size 
    \footnote{implemented with vector while depending on pointer to inner entry, which will be invalidate when the vector is resized.}
    is also changed. Besides, the Variable State Independent Decaying Sum (VSIDS) is also adopted
    in this milestone to provide more heuristic on branching from locality. However, for some test
    cases provided, the VSIDS leads to a worse performance, while the improvement of performance 
    gains from other test cases is not significant. Hence, this new feature is disabled by default.
    In the detail of implementation, this VSIDS also contains bias on the shorter clause, with adopted 
    the same heuristic in milestone 1.

  \section{Changes to Basic Structure}
    Most of the basic data structures remain unmodified since the last milestone, while the solver 
    state related data structures are changed to supprt the dynamic clause database. In last milestone, 
    clauses, assignment, literal watching status are all stored in fixed sized array. In this milestone,
    they are rebased on vector of pointer linking to dynamic allocated memory. Since the memory address are
    always the same regardless of the resize of the container, all of the data are used as pointer
    or iterator without dedundent copies. However, this implementation will cause more fragmented memory
    usage and increase the cache miss rate, which lead to some performance drop for small test cases used in
    milestone 1. 

    For the new features, a dedicated clause list in intruduced to manage different types of 
    clause seperately. This clause list is also prepared for further restriction and elimation of clause,
    which is planed to implement in next milestone.

    Some statistical counter is embedded into the solver to monitor the execution status, which can be toggled 
    via compiling options.

    Apart from the main solver program, another notable change is make to the parser to improve the compatibility for 
    data files.

  \section{Compiling Options and Usage}

    This section list some new options introduced in milestone 2.
    In the ``README.md'' file in the submission package, a more complete list of usage of the
    compiling options and executabe is presented.

    \subsection{Compiling options}

    This program is tesed in g++ 5.4.0 environment \footnote{on Ubuntu 16.04.1 platform} with ``-std=c++14'' option enabled (other options are not listed, which can refer to the Makefile).
    The compiling is controlled via Makefile, and tested ``make'' utility is GNU make
    \footnote{may fail to list all of the source in other implementations of make utility, such as the default one on some BSD platform.}.
    The Makefile can not only compile the program, but also execution the test script. The test is 
    done by use the SAT problem in the benchmark folder of the project. 

    The usage of the Makefile of this project can be summarized as following.
    \\ 

    \begin{tcolorbox}[breakable, blanker, width=\linewidth]
      \begin{minted}[autogobble, obeytabs, resetmargins, fontsize=\footnotesize]{shell}
        # cd to the projec directory before compiling

        # take defualt configuration
        make

        # set the option to non-empty value to enable 
        make FLAGS_DEBUG= FLAGS_COLOR=1      \
             FLAGS_VERBOSE=                  \
             FLAGS_PRINT_STATIS=1            \
             FLAGS_LITERAL_WEIGHT_DECAY=1    \
             FLAGS_LITERAL_WEIGHT_UPDATE=1 

        # test the program with diffenent set of problem
        make test-sanity
        make test-tiny
        make test-m2
        make test-m2-hard

        # use both the sanity and tiny problem
        make test
      \end{minted}
    \end{tcolorbox}

    The detail of the compiling options are listed as follows:
    \begin{itemize}
        \item {\bfseries FLAGS\_DEBUG}: toggle the debug mode. When enabled, the debug informations 
        will be embedded into the binary (via ``-ggdb3'' option) and will not enable higher level compilier optimization (via ``-O3'', etc.). Disabled by default.
        \item {\bfseries FLAGS\_COLOR}: toggle the color output. When enabled, the output message will be formatted with color in supported teriminals. However, when redirect the message to file for later check, the ANSI control sequence will be seen. In this scenario, diable the option or read output file via ``cat'' command. Enabled by default.
        \item {\bfseries FLAGS\_VERBOSE}: toggle the verbose mode. When enabled, the program will
        print detailed information about the execution procedure. This can dramatically decrease the performance of the problem and transform it from CPU bounded application to IO bounded application. This is useful to find out the source of error during debugging. Disabled by default.
        \item {\bfseries FLAGS\_PRINT\_STATIS}: toggle the print of statistic data. When enabled, 
        the program will print some statistic data of execution state every 3000 descstion. 
        Currently, the program will print the number of decision, the number of implication, 
        the number of conflict, the number of current learnt clause, and the maxium level of backtracking.
        \item {\bfseries FLAGS\_LITERAL\_WEIGHT\_DECAY}: toggle the literal weight decaying mode. When enabled,
        the weight of literal will decay as the excution of the solver. Current decaying strantegy is multiple $\frac{3}{4}$ every 200 decision.
        This option is independent with the following weight update option.
        \item {\bfseries FLAGS\_LITERAL\_WEIGHT\_UPDATE}: toggle the literal weight update mode. When enabled, 
        the weight of literal will be updated when new clause (the learnt conflict clause) is add into database.        
    \end{itemize}

    \subsection{Executable Usage}

    Owing to the limited number of features implemented at milestore one, the mere messsages
    which requires to be specified are the input SAT problem and output solution. The input can be 
    either a file or standard input from piped output of other program or mamual input, the latter can 
    be handy to perform some test. The output can be a file or standard output, similarly. The usage of the problem can be summarized as following.
    \\

    \begin{tcolorbox}[breakable, blanker, width=\linewidth]
      \begin{minted}[autogobble, obeytabs, resetmargins, fontsize=\footnotesize]{shell}
        # specify the input and output
        ./yasat [in-file|--stdin] [out-file|--stdout]

        # print help message
        ./yasat --help
      \end{minted}
    \end{tcolorbox}

    Currently, when use the ``-\--stdout'' option to print to result to screen, the message will still generate and will write to ``message.log'' file in the work directory during execution. 
    And parsing warning and error message will also be printed to standard output. To fully disable file writing, a feasible approach is 
    redirecting the output to ``null'' virtual device.

  \section{Difficulties and Problems}

  Based on the first milestone, this milestone is mainly to implement some important ``great ideas'' including the CDCL and VSIDS.
  As the growth of the code amount, it becomes harder and harder to trace and resolve misbehaviored program when some special 
  error can only be represented at large and complex test cases. Over 70\% of the time is spent on debugging. Another problem is 
  that the newly introduced features does not always direct to the improvement of performance. 

  For the given test case, there are still two of then cannot be solved within acceptable time, which may require some other techniques to resolve.
  

  \section{Next Milestone}

  For the milestone 3, some features are planed to implement including ``random restart'' and ``preprocessing''.
  Some other plan for the next milestone including optimization memory usage or allocation and add some application of SAT problem by encoding some practical problem.

  \end{multicols}

\end{document}
 